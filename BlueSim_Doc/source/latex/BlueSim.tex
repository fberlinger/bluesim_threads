%% Generated by Sphinx.
\def\sphinxdocclass{report}
\documentclass[letterpaper,10pt,english]{sphinxmanual}
\ifdefined\pdfpxdimen
   \let\sphinxpxdimen\pdfpxdimen\else\newdimen\sphinxpxdimen
\fi \sphinxpxdimen=.75bp\relax

\PassOptionsToPackage{warn}{textcomp}
\usepackage[utf8]{inputenc}
\ifdefined\DeclareUnicodeCharacter
% support both utf8 and utf8x syntaxes
\edef\sphinxdqmaybe{\ifdefined\DeclareUnicodeCharacterAsOptional\string"\fi}
  \DeclareUnicodeCharacter{\sphinxdqmaybe00A0}{\nobreakspace}
  \DeclareUnicodeCharacter{\sphinxdqmaybe2500}{\sphinxunichar{2500}}
  \DeclareUnicodeCharacter{\sphinxdqmaybe2502}{\sphinxunichar{2502}}
  \DeclareUnicodeCharacter{\sphinxdqmaybe2514}{\sphinxunichar{2514}}
  \DeclareUnicodeCharacter{\sphinxdqmaybe251C}{\sphinxunichar{251C}}
  \DeclareUnicodeCharacter{\sphinxdqmaybe2572}{\textbackslash}
\fi
\usepackage{cmap}
\usepackage[T1]{fontenc}
\usepackage{amsmath,amssymb,amstext}
\usepackage{babel}
\usepackage{times}
\usepackage[Bjarne]{fncychap}
\usepackage{sphinx}

\fvset{fontsize=\small}
\usepackage{geometry}

% Include hyperref last.
\usepackage{hyperref}
% Fix anchor placement for figures with captions.
\usepackage{hypcap}% it must be loaded after hyperref.
% Set up styles of URL: it should be placed after hyperref.
\urlstyle{same}
\addto\captionsenglish{\renewcommand{\contentsname}{Contents:}}

\addto\captionsenglish{\renewcommand{\figurename}{Fig.}}
\addto\captionsenglish{\renewcommand{\tablename}{Table}}
\addto\captionsenglish{\renewcommand{\literalblockname}{Listing}}

\addto\captionsenglish{\renewcommand{\literalblockcontinuedname}{continued from previous page}}
\addto\captionsenglish{\renewcommand{\literalblockcontinuesname}{continues on next page}}
\addto\captionsenglish{\renewcommand{\sphinxnonalphabeticalgroupname}{Non-alphabetical}}
\addto\captionsenglish{\renewcommand{\sphinxsymbolsname}{Symbols}}
\addto\captionsenglish{\renewcommand{\sphinxnumbersname}{Numbers}}

\addto\extrasenglish{\def\pageautorefname{page}}

\setcounter{tocdepth}{1}



\title{BlueSim Documentation}
\date{Nov 20, 2018}
\release{0.0.1}
\author{Florian Berlinger & Fritz Lekschas}
\newcommand{\sphinxlogo}{\vbox{}}
\renewcommand{\releasename}{Release}
\makeindex
\begin{document}

\pagestyle{empty}
\maketitle
\pagestyle{plain}
\sphinxtableofcontents
\pagestyle{normal}
\phantomsection\label{\detokenize{index::doc}}


BlueSim is a simulator for BlueBots.


\chapter{fish}
\label{\detokenize{index:module-fish}}\label{\detokenize{index:fish}}\index{fish (module)@\spxentry{fish}\spxextra{module}}\index{Fish (class in fish)@\spxentry{Fish}\spxextra{class in fish}}

\begin{fulllineitems}
\phantomsection\label{\detokenize{index:fish.Fish}}\pysiglinewithargsret{\sphinxbfcode{\sphinxupquote{class }}\sphinxcode{\sphinxupquote{fish.}}\sphinxbfcode{\sphinxupquote{Fish}}}{\emph{id, channel, interaction, lim\_neighbors={[}0, inf{]}, fish\_max\_speed=1, clock\_freq=1, neighbor\_weight=1.0, name='Unnamed', verbose=False}}{}
This class models each fish robot node in the network from the fish’
perspective.

Each fish has an ID, communicates over the channel, and perceives its
neighbors and takes actions accordingly. In taking actions, the fish can
weight information from neighbors based on their distance. The fish aims to
stay between a lower and upper limit of neighbors to maintain a cohesive
collective. It can move at a maximal speed and updates its behavior on
every clock tick.
\index{communicate() (fish.Fish method)@\spxentry{communicate()}\spxextra{fish.Fish method}}

\begin{fulllineitems}
\phantomsection\label{\detokenize{index:fish.Fish.communicate}}\pysiglinewithargsret{\sphinxbfcode{\sphinxupquote{communicate}}}{}{}
Broadcast all collected event messages.

This method is called as part of the second clock cycle.

\end{fulllineitems}

\index{comp\_center() (fish.Fish method)@\spxentry{comp\_center()}\spxextra{fish.Fish method}}

\begin{fulllineitems}
\phantomsection\label{\detokenize{index:fish.Fish.comp_center}}\pysiglinewithargsret{\sphinxbfcode{\sphinxupquote{comp\_center}}}{\emph{rel\_pos}}{}
Compute the (potentially weighted) centroid of the fish neighbors
\begin{description}
\item[{Arguments:}] \leavevmode\begin{description}
\item[{rel\_pos \{dict\} \textendash{} Dictionary of relative positions to the}] \leavevmode
neighboring fish.

\end{description}

\item[{Returns:}] \leavevmode
np.array \textendash{} 3D centroid

\end{description}

\end{fulllineitems}

\index{eval() (fish.Fish method)@\spxentry{eval()}\spxextra{fish.Fish method}}

\begin{fulllineitems}
\phantomsection\label{\detokenize{index:fish.Fish.eval}}\pysiglinewithargsret{\sphinxbfcode{\sphinxupquote{eval}}}{}{}
The fish evaluates its state

Currently the fish checks all responses to previous pings and evaluates
its relative position to all neighbors. Neighbors are other fish that
received the ping element.

\end{fulllineitems}

\index{homing\_handler() (fish.Fish method)@\spxentry{homing\_handler()}\spxextra{fish.Fish method}}

\begin{fulllineitems}
\phantomsection\label{\detokenize{index:fish.Fish.homing_handler}}\pysiglinewithargsret{\sphinxbfcode{\sphinxupquote{homing\_handler}}}{\emph{event}, \emph{pos}}{}
Homing handler, i.e., make fish aggregated extremely
\begin{description}
\item[{Arguments:}] \leavevmode
event \{Homing\} \textendash{} Homing event
pos \{np.array\} \textendash{} Position of the homing event initialtor

\end{description}

\end{fulllineitems}

\index{hop\_count\_handler() (fish.Fish method)@\spxentry{hop\_count\_handler()}\spxextra{fish.Fish method}}

\begin{fulllineitems}
\phantomsection\label{\detokenize{index:fish.Fish.hop_count_handler}}\pysiglinewithargsret{\sphinxbfcode{\sphinxupquote{hop\_count\_handler}}}{\emph{event}}{}
Hop count handler

Initialize only of the last hop count event is 4 clocks old. Otherwise
update the hop count and resend the new value only if its larger than
the previous hop count value.
\begin{description}
\item[{Arguments:}] \leavevmode
event \{HopCount\} \textendash{} Hop count event instance

\end{description}

\end{fulllineitems}

\index{info\_ext\_handler() (fish.Fish method)@\spxentry{info\_ext\_handler()}\spxextra{fish.Fish method}}

\begin{fulllineitems}
\phantomsection\label{\detokenize{index:fish.Fish.info_ext_handler}}\pysiglinewithargsret{\sphinxbfcode{\sphinxupquote{info\_ext\_handler}}}{\emph{event}}{}
External information handler

Always accept the external information and spread the news.
\begin{description}
\item[{Arguments:}] \leavevmode
event \{InfoExternal\} \textendash{} InfoExternal event

\end{description}

\end{fulllineitems}

\index{info\_int\_handler() (fish.Fish method)@\spxentry{info\_int\_handler()}\spxextra{fish.Fish method}}

\begin{fulllineitems}
\phantomsection\label{\detokenize{index:fish.Fish.info_int_handler}}\pysiglinewithargsret{\sphinxbfcode{\sphinxupquote{info\_int\_handler}}}{\emph{event}}{}
Internal information event handler.

Only accept the information of the clock is higher than from the last
information
\begin{description}
\item[{Arguments:}] \leavevmode
event \{InfoInternal\} \textendash{} Internal information event instance

\end{description}

\end{fulllineitems}

\index{leader\_election\_handler() (fish.Fish method)@\spxentry{leader\_election\_handler()}\spxextra{fish.Fish method}}

\begin{fulllineitems}
\phantomsection\label{\detokenize{index:fish.Fish.leader_election_handler}}\pysiglinewithargsret{\sphinxbfcode{\sphinxupquote{leader\_election\_handler}}}{\emph{event}}{}
Leader election handler
\begin{description}
\item[{Arguments:}] \leavevmode
event \{LeaderElection\} \textendash{} Leader election event instance

\end{description}

\end{fulllineitems}

\index{log() (fish.Fish method)@\spxentry{log()}\spxextra{fish.Fish method}}

\begin{fulllineitems}
\phantomsection\label{\detokenize{index:fish.Fish.log}}\pysiglinewithargsret{\sphinxbfcode{\sphinxupquote{log}}}{\emph{neighbors=\{\}}}{}
Log current state

\end{fulllineitems}

\index{move() (fish.Fish method)@\spxentry{move()}\spxextra{fish.Fish method}}

\begin{fulllineitems}
\phantomsection\label{\detokenize{index:fish.Fish.move}}\pysiglinewithargsret{\sphinxbfcode{\sphinxupquote{move}}}{\emph{neighbors}, \emph{rel\_pos}}{}
Make a cohesion and target-driven move

The move is determined by the relative position of the centroid and a
target position and is limited by the maximum fish speed.
\begin{description}
\item[{Arguments:}] \leavevmode\begin{description}
\item[{neighbors \{set\} \textendash{} Set of active neighbors, i.e., other fish that}] \leavevmode
responded to the most recent ping event.

\end{description}

rel\_pos \{dict\} \textendash{} Relative positions to all neighbors

\item[{Returns:}] \leavevmode
np.array \textendash{} Move direction as a 3D vector

\end{description}

\end{fulllineitems}

\index{move\_handler() (fish.Fish method)@\spxentry{move\_handler()}\spxextra{fish.Fish method}}

\begin{fulllineitems}
\phantomsection\label{\detokenize{index:fish.Fish.move_handler}}\pysiglinewithargsret{\sphinxbfcode{\sphinxupquote{move\_handler}}}{\emph{event}}{}
Handle move events, i.e., update the target position.
\begin{description}
\item[{Arguments:}] \leavevmode
event \{Move\} \textendash{} Event holding an x, y, and z target position

\end{description}

\end{fulllineitems}

\index{ping\_handler() (fish.Fish method)@\spxentry{ping\_handler()}\spxextra{fish.Fish method}}

\begin{fulllineitems}
\phantomsection\label{\detokenize{index:fish.Fish.ping_handler}}\pysiglinewithargsret{\sphinxbfcode{\sphinxupquote{ping\_handler}}}{\emph{neighbors}, \emph{rel\_pos}, \emph{event}}{}
Handle ping events

Adds the
\begin{description}
\item[{Arguments:}] \leavevmode\begin{description}
\item[{neighbors \{set\} \textendash{} Set of active neighbors, i.e., nodes from which}] \leavevmode
this fish received a ping event.

\item[{rel\_pos \{dict\} \textendash{} Dictionary of relative positions from this fish}] \leavevmode
to the source of the ping event.

\end{description}

event \{Ping\} \textendash{} The ping event instance

\end{description}

\end{fulllineitems}

\index{run() (fish.Fish method)@\spxentry{run()}\spxextra{fish.Fish method}}

\begin{fulllineitems}
\phantomsection\label{\detokenize{index:fish.Fish.run}}\pysiglinewithargsret{\sphinxbfcode{\sphinxupquote{run}}}{}{}
Run the process recursively

This method simulates the fish and calls \sphinxtitleref{eval} on every clock tick as
long as the fish \sphinxtitleref{is\_started}.

\end{fulllineitems}

\index{start() (fish.Fish method)@\spxentry{start()}\spxextra{fish.Fish method}}

\begin{fulllineitems}
\phantomsection\label{\detokenize{index:fish.Fish.start}}\pysiglinewithargsret{\sphinxbfcode{\sphinxupquote{start}}}{}{}
Start the process

This sets \sphinxtitleref{is\_started} to true and invokes \sphinxtitleref{run()}.

\end{fulllineitems}

\index{start\_hop\_count\_handler() (fish.Fish method)@\spxentry{start\_hop\_count\_handler()}\spxextra{fish.Fish method}}

\begin{fulllineitems}
\phantomsection\label{\detokenize{index:fish.Fish.start_hop_count_handler}}\pysiglinewithargsret{\sphinxbfcode{\sphinxupquote{start\_hop\_count\_handler}}}{\emph{event}}{}
Hop count start handler

Always accept a new start event for a hop count
\begin{description}
\item[{Arguments:}] \leavevmode
event \{StartHopCount\} \textendash{} Hop count start event

\end{description}

\end{fulllineitems}

\index{start\_leader\_election\_handler() (fish.Fish method)@\spxentry{start\_leader\_election\_handler()}\spxextra{fish.Fish method}}

\begin{fulllineitems}
\phantomsection\label{\detokenize{index:fish.Fish.start_leader_election_handler}}\pysiglinewithargsret{\sphinxbfcode{\sphinxupquote{start\_leader\_election\_handler}}}{\emph{event}}{}
Leader election start handler

Always accept a new start event for a leader election
\begin{description}
\item[{Arguments:}] \leavevmode
event \{StartLeaderElection\} \textendash{} Leader election start event

\end{description}

\end{fulllineitems}

\index{stop() (fish.Fish method)@\spxentry{stop()}\spxextra{fish.Fish method}}

\begin{fulllineitems}
\phantomsection\label{\detokenize{index:fish.Fish.stop}}\pysiglinewithargsret{\sphinxbfcode{\sphinxupquote{stop}}}{}{}
Stop the process

This sets \sphinxtitleref{is\_started} to false.

\end{fulllineitems}

\index{update\_behavior() (fish.Fish method)@\spxentry{update\_behavior()}\spxextra{fish.Fish method}}

\begin{fulllineitems}
\phantomsection\label{\detokenize{index:fish.Fish.update_behavior}}\pysiglinewithargsret{\sphinxbfcode{\sphinxupquote{update\_behavior}}}{}{}
Update the fish behavior.

This actively changes the cohesion strategy to either ‘wait’, i.e, do
not care about any neighbors or ‘signal\_aircraft’, i.e., aggregate with
as many fish friends as possible.

In robotics ‘signal\_aircraft’ is a secret key word for robo-fish-nerds
to gather in a secret lab until some robo fish finds a robo aircraft.

\end{fulllineitems}

\index{weight\_neighbor() (fish.Fish method)@\spxentry{weight\_neighbor()}\spxextra{fish.Fish method}}

\begin{fulllineitems}
\phantomsection\label{\detokenize{index:fish.Fish.weight_neighbor}}\pysiglinewithargsret{\sphinxbfcode{\sphinxupquote{weight\_neighbor}}}{\emph{rel\_pos\_to\_neighbor}}{}
Weight neighbors by the relative position to them

Currently only returns a static value but this could be tweaked in the
future to calculate a weighted center point.
\begin{description}
\item[{Arguments:}] \leavevmode
rel\_pos\_to\_neighbor \{np.array\} \textendash{} Relative position to a neighbor

\item[{Returns:}] \leavevmode
float \textendash{} Weight for this neighbor

\end{description}

\end{fulllineitems}


\end{fulllineitems}



\chapter{environment}
\label{\detokenize{index:module-environment}}\label{\detokenize{index:environment}}\index{environment (module)@\spxentry{environment}\spxextra{module}}\index{Environment (class in environment)@\spxentry{Environment}\spxextra{class in environment}}

\begin{fulllineitems}
\phantomsection\label{\detokenize{index:environment.Environment}}\pysiglinewithargsret{\sphinxbfcode{\sphinxupquote{class }}\sphinxcode{\sphinxupquote{environment.}}\sphinxbfcode{\sphinxupquote{Environment}}}{\emph{node\_pos}, \emph{distortion}, \emph{prob\_type='quadratic'}, \emph{conn\_thres=inf}, \emph{conn\_drop=1}, \emph{noise\_magnitude=0.1}, \emph{verbose=False}}{}
The dynamic network of robot nodes in the underwater environment

This class keeps track of the network dynamics by storing the positions of
all nodes. It contains functions to derive the distorted position from a
target position by adding a distortion and noise, to update the position of
a node, to update the distance between nodes, to derive the probability of
receiving a message from a node based on that distance, and to get the
relative position from one node to another node.
\index{get\_distorted\_pos() (environment.Environment method)@\spxentry{get\_distorted\_pos()}\spxextra{environment.Environment method}}

\begin{fulllineitems}
\phantomsection\label{\detokenize{index:environment.Environment.get_distorted_pos}}\pysiglinewithargsret{\sphinxbfcode{\sphinxupquote{get\_distorted\_pos}}}{\emph{source\_index}, \emph{target\_pos}}{}
Calculate the distorted target position of a node.

This method adds random noise and the position-based distortion onto
the ideal target position to calculate the final position of the node.
\begin{description}
\item[{Arguments:}] \leavevmode\begin{description}
\item[{source\_index \{int\} \textendash{} Index of the source node which position is to}] \leavevmode
be distorted.

\end{description}

target\_pos \{np.array\} \textendash{} Ideal target position to be distorted

\item[{Returns:}] \leavevmode
np.array \textendash{} Final position of the node.

\end{description}

\end{fulllineitems}

\index{get\_rel\_pos() (environment.Environment method)@\spxentry{get\_rel\_pos()}\spxextra{environment.Environment method}}

\begin{fulllineitems}
\phantomsection\label{\detokenize{index:environment.Environment.get_rel_pos}}\pysiglinewithargsret{\sphinxbfcode{\sphinxupquote{get\_rel\_pos}}}{\emph{source\_index}, \emph{target\_index}}{}
Calculate the relative position of two nodes

Calculate the vector pointing from the source node to the target node.
\begin{description}
\item[{Arguments:}] \leavevmode\begin{description}
\item[{source\_index \{int\} \textendash{} Index of the source node, i.e., the node for}] \leavevmode
which the relative position to target is specified.

\item[{target\_index \{int\} \textendash{} Index of the target node, i.e., the node to}] \leavevmode
which source is relatively positioned to.

\end{description}

\item[{Returns:}] \leavevmode
np,array \textendash{} Vector pointing from source to target

\end{description}

\end{fulllineitems}

\index{prob() (environment.Environment method)@\spxentry{prob()}\spxextra{environment.Environment method}}

\begin{fulllineitems}
\phantomsection\label{\detokenize{index:environment.Environment.prob}}\pysiglinewithargsret{\sphinxbfcode{\sphinxupquote{prob}}}{\emph{node\_a\_index}, \emph{node\_b\_index}}{}
Calculate the probability of connectivity of two points based on
their Eucledian distance.
\begin{description}
\item[{Arguments:}] \leavevmode
node\_a\_index \{int\} \textendash{} Node A index
node\_b\_index \{int\} \textendash{} Node B index

\item[{Returns:}] \leavevmode
float \textendash{} probability of connectivity

\end{description}

\end{fulllineitems}

\index{prob\_binary() (environment.Environment method)@\spxentry{prob\_binary()}\spxextra{environment.Environment method}}

\begin{fulllineitems}
\phantomsection\label{\detokenize{index:environment.Environment.prob_binary}}\pysiglinewithargsret{\sphinxbfcode{\sphinxupquote{prob\_binary}}}{\emph{distance}}{}
Simulate binary connectivity probability

This function either returns 1 or 0 if the distance of two nodes is
smaller (or larger) than the user defined threshold.
\begin{description}
\item[{Arguments:}] \leavevmode
distance \{float\} \textendash{} Eucledian distance

\item[{Returns:}] \leavevmode\begin{description}
\item[{float \textendash{} probability of connectivity. The probability is either 1}] \leavevmode
or 0 depending on the distance threshold.

\end{description}

\end{description}

\end{fulllineitems}

\index{prob\_dist() (environment.Environment method)@\spxentry{prob\_dist()}\spxextra{environment.Environment method}}

\begin{fulllineitems}
\phantomsection\label{\detokenize{index:environment.Environment.prob_dist}}\pysiglinewithargsret{\sphinxbfcode{\sphinxupquote{prob\_dist}}}{\emph{distance}}{}
Calls the approriate probability functions

The returned probability depends on prob\_type
\begin{description}
\item[{Arguments:}] \leavevmode
distance \{float\} \textendash{} Eucledian distance

\item[{Returns:}] \leavevmode
float \textendash{} probability of connectivity

\end{description}

\end{fulllineitems}

\index{prob\_quadratic() (environment.Environment method)@\spxentry{prob\_quadratic()}\spxextra{environment.Environment method}}

\begin{fulllineitems}
\phantomsection\label{\detokenize{index:environment.Environment.prob_quadratic}}\pysiglinewithargsret{\sphinxbfcode{\sphinxupquote{prob\_quadratic}}}{\emph{distance}}{}
Simulate quadradic connectivity probability
\begin{description}
\item[{Arguments:}] \leavevmode
distance \{float\} \textendash{} Eucledian distance

\item[{Returns:}] \leavevmode\begin{description}
\item[{float \textendash{} probability of connectivity as a function of the distance.}] \leavevmode
The probability drops quadratically.

\end{description}

\end{description}

\end{fulllineitems}

\index{prob\_sigmoid() (environment.Environment method)@\spxentry{prob\_sigmoid()}\spxextra{environment.Environment method}}

\begin{fulllineitems}
\phantomsection\label{\detokenize{index:environment.Environment.prob_sigmoid}}\pysiglinewithargsret{\sphinxbfcode{\sphinxupquote{prob\_sigmoid}}}{\emph{distance}}{}
Simulate sigmoid connectivity probability
\begin{description}
\item[{Arguments:}] \leavevmode
distance \{float\} \textendash{} Eucledian distance

\item[{Returns:}] \leavevmode\begin{description}
\item[{float \textendash{} probability of connectivity as a sigmoid function of the}] \leavevmode
distance.

\end{description}

\end{description}

\end{fulllineitems}

\index{set\_pos() (environment.Environment method)@\spxentry{set\_pos()}\spxextra{environment.Environment method}}

\begin{fulllineitems}
\phantomsection\label{\detokenize{index:environment.Environment.set_pos}}\pysiglinewithargsret{\sphinxbfcode{\sphinxupquote{set\_pos}}}{\emph{source\_index}, \emph{new\_pos}}{}
Set the new position

Save the new position into the positions array.
\begin{description}
\item[{Arguments:}] \leavevmode
source\_index \{int\} \textendash{} Index of the node position to be set
new\_pos \{np.array\} \textendash{} New node position ({[}x, y, z{]}) to be set.

\end{description}

\end{fulllineitems}

\index{update\_distance() (environment.Environment method)@\spxentry{update\_distance()}\spxextra{environment.Environment method}}

\begin{fulllineitems}
\phantomsection\label{\detokenize{index:environment.Environment.update_distance}}\pysiglinewithargsret{\sphinxbfcode{\sphinxupquote{update\_distance}}}{}{}
Calculate pairwise distances of every node

Calculate and saves the pairwise distance of every node.

\end{fulllineitems}


\end{fulllineitems}



\chapter{interaction}
\label{\detokenize{index:module-interaction}}\label{\detokenize{index:interaction}}\index{interaction (module)@\spxentry{interaction}\spxextra{module}}\index{Interaction (class in interaction)@\spxentry{Interaction}\spxextra{class in interaction}}

\begin{fulllineitems}
\phantomsection\label{\detokenize{index:interaction.Interaction}}\pysiglinewithargsret{\sphinxbfcode{\sphinxupquote{class }}\sphinxcode{\sphinxupquote{interaction.}}\sphinxbfcode{\sphinxupquote{Interaction}}}{\emph{environment}, \emph{verbose=False}}{}
Underwater interactions

This class models interactions of the fish with their environment, e.g.,
to perceive other fish or to change their position.
\index{move() (interaction.Interaction method)@\spxentry{move()}\spxextra{interaction.Interaction method}}

\begin{fulllineitems}
\phantomsection\label{\detokenize{index:interaction.Interaction.move}}\pysiglinewithargsret{\sphinxbfcode{\sphinxupquote{move}}}{\emph{source\_id}, \emph{target\_direction}}{}
Move a fish

Moves the fish relatively into the given direction and adds
target-based distortion to the fish position.
\begin{description}
\item[{Arguments:}] \leavevmode
source\_id \{int\} \textendash{} Fish identifier
target\_direction \{np.array\} \textendash{} Relative direction to move to

\end{description}

\end{fulllineitems}

\index{perceive\_object() (interaction.Interaction method)@\spxentry{perceive\_object()}\spxextra{interaction.Interaction method}}

\begin{fulllineitems}
\phantomsection\label{\detokenize{index:interaction.Interaction.perceive_object}}\pysiglinewithargsret{\sphinxbfcode{\sphinxupquote{perceive\_object}}}{\emph{source\_id}, \emph{pos}}{}
Perceive the relative position to an object

This simulates the fish’s perception of external sources and targets.
\begin{description}
\item[{Arguments:}] \leavevmode\begin{description}
\item[{source\_id \{int\} \textendash{} Index of the fish that wants to know its}] \leavevmode
location

\end{description}

pos \{np.array\} \textendash{} X, Y, and Z position of the object

\end{description}

\end{fulllineitems}

\index{perceive\_pos() (interaction.Interaction method)@\spxentry{perceive\_pos()}\spxextra{interaction.Interaction method}}

\begin{fulllineitems}
\phantomsection\label{\detokenize{index:interaction.Interaction.perceive_pos}}\pysiglinewithargsret{\sphinxbfcode{\sphinxupquote{perceive\_pos}}}{\emph{source\_id}, \emph{target\_id}}{}
Perceive the relative position to another fish

This simulates the fish’s perception of neighbors.
\begin{description}
\item[{Arguments:}] \leavevmode
source\_id \{int\} \textendash{} Index of the fish to be perceived
target\_id \{int\} \textendash{} Index of the fish to be perceived

\end{description}

\end{fulllineitems}


\end{fulllineitems}



\chapter{channel}
\label{\detokenize{index:module-channel}}\label{\detokenize{index:channel}}\index{channel (module)@\spxentry{channel}\spxextra{module}}\index{Channel (class in channel)@\spxentry{Channel}\spxextra{class in channel}}

\begin{fulllineitems}
\phantomsection\label{\detokenize{index:channel.Channel}}\pysiglinewithargsret{\sphinxbfcode{\sphinxupquote{class }}\sphinxcode{\sphinxupquote{channel.}}\sphinxbfcode{\sphinxupquote{Channel}}}{\emph{environment}, \emph{verbose=False}}{}
Underwater wireless communication channel

This class models the underwater communication between fish instances and
connects fish to the environmental network.
\index{intercept() (channel.Channel method)@\spxentry{intercept()}\spxextra{channel.Channel method}}

\begin{fulllineitems}
\phantomsection\label{\detokenize{index:channel.Channel.intercept}}\pysiglinewithargsret{\sphinxbfcode{\sphinxupquote{intercept}}}{\emph{observer}}{}
Let an observer intercept all messages.

It’s really unfortunate but there are not just holes in Swiss cheese.
Our channel is no exception and a god-like observer is able to listen
to all transmitted messages in the name of research. Please don’t tell
anyone.
\begin{description}
\item[{Arguments:}] \leavevmode
observer \{Observer\} \textendash{} The all mighty observer

\end{description}

\end{fulllineitems}

\index{set\_nodes() (channel.Channel method)@\spxentry{set\_nodes()}\spxextra{channel.Channel method}}

\begin{fulllineitems}
\phantomsection\label{\detokenize{index:channel.Channel.set_nodes}}\pysiglinewithargsret{\sphinxbfcode{\sphinxupquote{set\_nodes}}}{\emph{nodes}}{}
This method just stores a references to all nodes
\begin{description}
\item[{Arguments:}] \leavevmode
nodes \{list\} \textendash{} List of node instances

\end{description}

\end{fulllineitems}

\index{transmit() (channel.Channel method)@\spxentry{transmit()}\spxextra{channel.Channel method}}

\begin{fulllineitems}
\phantomsection\label{\detokenize{index:channel.Channel.transmit}}\pysiglinewithargsret{\sphinxbfcode{\sphinxupquote{transmit}}}{\emph{source}, \emph{event}, \emph{pos=array({[}0.}, \emph{0.}, \emph{0.{]})}, \emph{is\_observer=False}}{}
Transmit a broadcasted event to node instances

This method gets the probability of connectedness between two nodes
from the environment and adds the events on the node instances given
that probability.
\begin{description}
\item[{Arguments:}] \leavevmode
source \{*\} \textendash{} Node instance
event \{Event\} \textendash{} Some event to be broadcasted

\end{description}

\end{fulllineitems}


\end{fulllineitems}



\chapter{observer}
\label{\detokenize{index:module-observer}}\label{\detokenize{index:observer}}\index{observer (module)@\spxentry{observer}\spxextra{module}}\index{Observer (class in observer)@\spxentry{Observer}\spxextra{class in observer}}

\begin{fulllineitems}
\phantomsection\label{\detokenize{index:observer.Observer}}\pysiglinewithargsret{\sphinxbfcode{\sphinxupquote{class }}\sphinxcode{\sphinxupquote{observer.}}\sphinxbfcode{\sphinxupquote{Observer}}}{\emph{environment}, \emph{fish}, \emph{channel}, \emph{clock\_freq=1}, \emph{fish\_pos=None}, \emph{verbose=False}}{}
The god-like observer keeps track of the fish movement for analysis.
\index{activate\_reset() (observer.Observer method)@\spxentry{activate\_reset()}\spxextra{observer.Observer method}}

\begin{fulllineitems}
\phantomsection\label{\detokenize{index:observer.Observer.activate_reset}}\pysiglinewithargsret{\sphinxbfcode{\sphinxupquote{activate\_reset}}}{}{}
Activate automatic resetting of the fish positions on a new
instruction.

\end{fulllineitems}

\index{check\_info\_consistency() (observer.Observer method)@\spxentry{check\_info\_consistency()}\spxextra{observer.Observer method}}

\begin{fulllineitems}
\phantomsection\label{\detokenize{index:observer.Observer.check_info_consistency}}\pysiglinewithargsret{\sphinxbfcode{\sphinxupquote{check\_info\_consistency}}}{}{}
Check consistency of a tracked information

\end{fulllineitems}

\index{check\_instructions() (observer.Observer method)@\spxentry{check\_instructions()}\spxextra{observer.Observer method}}

\begin{fulllineitems}
\phantomsection\label{\detokenize{index:observer.Observer.check_instructions}}\pysiglinewithargsret{\sphinxbfcode{\sphinxupquote{check\_instructions}}}{}{}
Check external instructions to be broadcasted.

If we reach the clock cycle in which they should be broadcasted, send
them out.

\end{fulllineitems}

\index{check\_transmissions() (observer.Observer method)@\spxentry{check\_transmissions()}\spxextra{observer.Observer method}}

\begin{fulllineitems}
\phantomsection\label{\detokenize{index:observer.Observer.check_transmissions}}\pysiglinewithargsret{\sphinxbfcode{\sphinxupquote{check\_transmissions}}}{}{}
Check intercepted transmission from the channel

\end{fulllineitems}

\index{deactivate\_reset() (observer.Observer method)@\spxentry{deactivate\_reset()}\spxextra{observer.Observer method}}

\begin{fulllineitems}
\phantomsection\label{\detokenize{index:observer.Observer.deactivate_reset}}\pysiglinewithargsret{\sphinxbfcode{\sphinxupquote{deactivate\_reset}}}{}{}
Deactivate automatic resetting of the fish positions on a new
instruction.

\end{fulllineitems}

\index{eval() (observer.Observer method)@\spxentry{eval()}\spxextra{observer.Observer method}}

\begin{fulllineitems}
\phantomsection\label{\detokenize{index:observer.Observer.eval}}\pysiglinewithargsret{\sphinxbfcode{\sphinxupquote{eval}}}{}{}
Save the position and connectivity status of the fish.

\end{fulllineitems}

\index{instruct() (observer.Observer method)@\spxentry{instruct()}\spxextra{observer.Observer method}}

\begin{fulllineitems}
\phantomsection\label{\detokenize{index:observer.Observer.instruct}}\pysiglinewithargsret{\sphinxbfcode{\sphinxupquote{instruct}}}{\emph{event}, \emph{rel\_clock=0}, \emph{fish\_id=None}, \emph{pos=array({[}0.}, \emph{0.}, \emph{0.{]})}, \emph{fish\_all=False}}{}
Make the observer instruct the fish swarm.

This will effectively trigger an event in the fish environment, like an
instruction or some kind of obstacle.
\begin{description}
\item[{Arguments:}] \leavevmode
event \{*\} \textendash{} Some event instance.

\item[{Keyword Arguments:}] \leavevmode\begin{description}
\item[{rel\_clock \{number\} \textendash{} Number of relative clock cycles from now when}] \leavevmode
to broadcast the event (default: \{0\})

\item[{fish\_id \{int\} \textendash{} If not \sphinxtitleref{None} directly put the event on the}] \leavevmode
fish with this id. (default: \{None\})

\item[{pos \{np.array\} \textendash{} Imaginary event position. Used to determine the}] \leavevmode
probability that fish will hear the
event. (default: \{np.zeros(2,)\})

\item[{fish\_all \{bool\} \textendash{} If \sphinxtitleref{true} all fish will immediately receive the}] \leavevmode
event, i.e., no probabilistic event anymore. (default: \{False\})

\end{description}

\end{description}

\end{fulllineitems}

\index{plot() (observer.Observer method)@\spxentry{plot()}\spxextra{observer.Observer method}}

\begin{fulllineitems}
\phantomsection\label{\detokenize{index:observer.Observer.plot}}\pysiglinewithargsret{\sphinxbfcode{\sphinxupquote{plot}}}{\emph{dark=False}, \emph{white\_axis=False}, \emph{no\_legend=False}, \emph{show\_bar\_chart=False}, \emph{no\_star=False}}{}
Plot the fish movement

\end{fulllineitems}

\index{run() (observer.Observer method)@\spxentry{run()}\spxextra{observer.Observer method}}

\begin{fulllineitems}
\phantomsection\label{\detokenize{index:observer.Observer.run}}\pysiglinewithargsret{\sphinxbfcode{\sphinxupquote{run}}}{}{}
Run the process recursively

This method simulates the fish and calls \sphinxtitleref{eval} on every clock tick as
long as the fish \sphinxtitleref{is\_started}.

\end{fulllineitems}

\index{start() (observer.Observer method)@\spxentry{start()}\spxextra{observer.Observer method}}

\begin{fulllineitems}
\phantomsection\label{\detokenize{index:observer.Observer.start}}\pysiglinewithargsret{\sphinxbfcode{\sphinxupquote{start}}}{}{}
Start the process

This sets \sphinxtitleref{is\_started} to true and invokes \sphinxtitleref{run()}.

\end{fulllineitems}

\index{stop() (observer.Observer method)@\spxentry{stop()}\spxextra{observer.Observer method}}

\begin{fulllineitems}
\phantomsection\label{\detokenize{index:observer.Observer.stop}}\pysiglinewithargsret{\sphinxbfcode{\sphinxupquote{stop}}}{}{}
Stop the process

This sets \sphinxtitleref{is\_started} to false.

\end{fulllineitems}


\end{fulllineitems}



\chapter{events}
\label{\detokenize{index:module-events}}\label{\detokenize{index:events}}\index{events (module)@\spxentry{events}\spxextra{module}}\index{Homing (class in events)@\spxentry{Homing}\spxextra{class in events}}

\begin{fulllineitems}
\phantomsection\label{\detokenize{index:events.Homing}}\pysigline{\sphinxbfcode{\sphinxupquote{class }}\sphinxcode{\sphinxupquote{events.}}\sphinxbfcode{\sphinxupquote{Homing}}}
Homing towards an external source

\end{fulllineitems}

\index{HopCount (class in events)@\spxentry{HopCount}\spxextra{class in events}}

\begin{fulllineitems}
\phantomsection\label{\detokenize{index:events.HopCount}}\pysiglinewithargsret{\sphinxbfcode{\sphinxupquote{class }}\sphinxcode{\sphinxupquote{events.}}\sphinxbfcode{\sphinxupquote{HopCount}}}{\emph{id}, \emph{clock}, \emph{hops=0}}{}
Broadcast hop counts

A funny side note: in Germany distributed and DNA-based organisms (often
called humans) shout “Hop Hop rin in Kopp”, which is a similar but slightly
different event type that makes other human instances to instantly enjoy
a whole glass of juicy beer in just a single hop! Highly efficient!

\end{fulllineitems}

\index{InfoExternal (class in events)@\spxentry{InfoExternal}\spxextra{class in events}}

\begin{fulllineitems}
\phantomsection\label{\detokenize{index:events.InfoExternal}}\pysiglinewithargsret{\sphinxbfcode{\sphinxupquote{class }}\sphinxcode{\sphinxupquote{events.}}\sphinxbfcode{\sphinxupquote{InfoExternal}}}{\emph{message}, \emph{track=False}}{}
Share external information with fish

\end{fulllineitems}

\index{InfoInternal (class in events)@\spxentry{InfoInternal}\spxextra{class in events}}

\begin{fulllineitems}
\phantomsection\label{\detokenize{index:events.InfoInternal}}\pysiglinewithargsret{\sphinxbfcode{\sphinxupquote{class }}\sphinxcode{\sphinxupquote{events.}}\sphinxbfcode{\sphinxupquote{InfoInternal}}}{\emph{id}, \emph{clock}, \emph{message}, \emph{hops=0}}{}
Share information internally with other fish

\end{fulllineitems}

\index{LeaderElection (class in events)@\spxentry{LeaderElection}\spxextra{class in events}}

\begin{fulllineitems}
\phantomsection\label{\detokenize{index:events.LeaderElection}}\pysiglinewithargsret{\sphinxbfcode{\sphinxupquote{class }}\sphinxcode{\sphinxupquote{events.}}\sphinxbfcode{\sphinxupquote{LeaderElection}}}{\emph{id}, \emph{max\_id}}{}
Broadcast a leader election

\end{fulllineitems}

\index{Move (class in events)@\spxentry{Move}\spxextra{class in events}}

\begin{fulllineitems}
\phantomsection\label{\detokenize{index:events.Move}}\pysiglinewithargsret{\sphinxbfcode{\sphinxupquote{class }}\sphinxcode{\sphinxupquote{events.}}\sphinxbfcode{\sphinxupquote{Move}}}{\emph{x=0}, \emph{y=0}, \emph{z=0}}{}
Make the fish move to a target direction

\end{fulllineitems}

\index{Ping (class in events)@\spxentry{Ping}\spxextra{class in events}}

\begin{fulllineitems}
\phantomsection\label{\detokenize{index:events.Ping}}\pysiglinewithargsret{\sphinxbfcode{\sphinxupquote{class }}\sphinxcode{\sphinxupquote{events.}}\sphinxbfcode{\sphinxupquote{Ping}}}{\emph{id}}{}
Ping your beloved neighbor fish

\end{fulllineitems}

\index{StartHopCount (class in events)@\spxentry{StartHopCount}\spxextra{class in events}}

\begin{fulllineitems}
\phantomsection\label{\detokenize{index:events.StartHopCount}}\pysigline{\sphinxbfcode{\sphinxupquote{class }}\sphinxcode{\sphinxupquote{events.}}\sphinxbfcode{\sphinxupquote{StartHopCount}}}
Initialize a hop count.

\end{fulllineitems}

\index{StartLeaderElection (class in events)@\spxentry{StartLeaderElection}\spxextra{class in events}}

\begin{fulllineitems}
\phantomsection\label{\detokenize{index:events.StartLeaderElection}}\pysigline{\sphinxbfcode{\sphinxupquote{class }}\sphinxcode{\sphinxupquote{events.}}\sphinxbfcode{\sphinxupquote{StartLeaderElection}}}
Initialize a leader election

\end{fulllineitems}



\chapter{eventcodes}
\label{\detokenize{index:module-eventcodes}}\label{\detokenize{index:eventcodes}}\index{eventcodes (module)@\spxentry{eventcodes}\spxextra{module}}

\chapter{utils}
\label{\detokenize{index:module-utils}}\label{\detokenize{index:utils}}\index{utils (module)@\spxentry{utils}\spxextra{module}}
Helper methods to run the experiment
\index{generate\_distortion() (in module utils)@\spxentry{generate\_distortion()}\spxextra{in module utils}}

\begin{fulllineitems}
\phantomsection\label{\detokenize{index:utils.generate_distortion}}\pysiglinewithargsret{\sphinxcode{\sphinxupquote{utils.}}\sphinxbfcode{\sphinxupquote{generate\_distortion}}}{\emph{type='linear'}, \emph{magnitude=1}, \emph{n=10}, \emph{show=False}}{}
Generates a distortion model represented as a vector field

\end{fulllineitems}

\index{generate\_fish() (in module utils)@\spxentry{generate\_fish()}\spxextra{in module utils}}

\begin{fulllineitems}
\phantomsection\label{\detokenize{index:utils.generate_fish}}\pysiglinewithargsret{\sphinxcode{\sphinxupquote{utils.}}\sphinxbfcode{\sphinxupquote{generate\_fish}}}{\emph{n}, \emph{channel}, \emph{interaction}, \emph{lim\_neighbors}, \emph{neighbor\_weights=None}, \emph{fish\_max\_speeds=None}, \emph{clock\_freqs=None}, \emph{verbose=False}, \emph{names=None}}{}
Generate some fish
\begin{description}
\item[{Arguments:}] \leavevmode
n \{int\} \textendash{} Number of fish to generate
channel \{Channel\} \textendash{} Channel instance
interaction \{Interaction\} \textendash{} Interaction instance
lim\_neighbors \{list\} \textendash{} Tuple of min and max neighbors
neighbor\_weight \{float\textbar{}list\} \textendash{} List of neighbor weights
fish\_max\_speeds \{float\textbar{}list\} \textendash{} List of max speeds
clock\_freqs \{int\textbar{}list\} \textendash{} List of clock speeds
names \{list\} \textendash{} List of names for your fish

\end{description}

\end{fulllineitems}

\index{init\_simulation() (in module utils)@\spxentry{init\_simulation()}\spxextra{in module utils}}

\begin{fulllineitems}
\phantomsection\label{\detokenize{index:utils.init_simulation}}\pysiglinewithargsret{\sphinxcode{\sphinxupquote{utils.}}\sphinxbfcode{\sphinxupquote{init\_simulation}}}{\emph{clock\_freq}, \emph{single\_time}, \emph{offset\_time}, \emph{num\_trials}, \emph{final\_buffer}, \emph{run\_time}, \emph{num\_fish}, \emph{size\_dist}, \emph{center}, \emph{spread}, \emph{fish\_pos}, \emph{lim\_neighbors}, \emph{neighbor\_weights}, \emph{fish\_max\_speeds}, \emph{noise\_magnitude}, \emph{conn\_thres}, \emph{prob\_type}, \emph{dist\_type}, \emph{verbose}, \emph{conn\_drop=1.0}}{}
Initialize all the instances needed for a simulation
\begin{description}
\item[{Arguments:}] \leavevmode
clock\_freq \{int\} \textendash{} Clock frequency for each fish.
single\_time \{float\} \textendash{} Number clock cycles per individual run.
offset\_time \{float\} \textendash{} Initial clock offset time
num\_trials \{int\} \textendash{} Number of trials per experiment.
final\_buffer \{float\} \textendash{} Final clock buffer (because the clocks don’t
\begin{quote}

sync perfectly).
\end{quote}

run\_time \{float\} \textendash{} Total run time in seconds.
num\_fish \{int\} \textendash{} Number of fish.
size\_dist \{int\} \textendash{} Distortion field size.
center \{float\} \textendash{} Distortion field center.
spread \{float\} \textendash{} Initial fish position spread.
fish\_pos \{np.array\} \textendash{} Initial fish position.
lim\_neighbors \{list\} \textendash{} Min. and max. desired neighbors. If too few
\begin{quote}

neighbors start aggregation, if too many neighbors disperse!
\end{quote}

neighbor\_weights \{float\} \textendash{} Distance-depending neighbor weight.
fish\_max\_speeds \{float\} \textendash{} Max fish speed.
noise\_magnitude \{float\} \textendash{} Amount of white noise added to each move.
conn\_thres \{float\} \textendash{} Distance at which the connection either cuts off
\begin{quote}

or starts dropping severely.
\end{quote}
\begin{description}
\item[{prob\_type \{str\} \textendash{} Probability type. Can be \sphinxtitleref{binary}, \sphinxtitleref{quadratic}, or}] \leavevmode
\sphinxtitleref{sigmoid}.

\end{description}

dist\_type \{str\} \textendash{} Position distortion type
verbose \{bool\} \textendash{} If \sphinxtitleref{true} print a lot of stuff

\item[{Keyword Arguments:}] \leavevmode\begin{description}
\item[{conn\_drop \{number\} \textendash{} Defined the connection drop for the sigmoid}] \leavevmode
(default: \{1.0\})

\end{description}

\item[{Returns:}] \leavevmode\begin{description}
\item[{tuple \textendash{} Quintuple holding the \sphinxtitleref{channel}, \sphinxtitleref{environment}, {\color{red}\bfseries{}{}`}fish,}] \leavevmode
\sphinxtitleref{interaction}, and \sphinxtitleref{observer}

\end{description}

\end{description}

\end{fulllineitems}

\index{run\_simulation() (in module utils)@\spxentry{run\_simulation()}\spxextra{in module utils}}

\begin{fulllineitems}
\phantomsection\label{\detokenize{index:utils.run_simulation}}\pysiglinewithargsret{\sphinxcode{\sphinxupquote{utils.}}\sphinxbfcode{\sphinxupquote{run\_simulation}}}{\emph{fish}, \emph{observer}, \emph{run\_time=10}, \emph{dark=False}, \emph{white\_axis=False}, \emph{no\_legend=False}, \emph{no\_star=False}}{}
Start the simulation.
\begin{description}
\item[{Arguments:}] \leavevmode
fish \{list\} \textendash{} List of fish instances
observer \{Observer\} \textendash{} Observer instance

\item[{Keyword Arguments:}] \leavevmode
run\_time \{number\} \textendash{} Total run time in seconds (default: \{10\})
dark \{bool\} \textendash{} If \sphinxtitleref{True} plot a dark chart (default: \{False\})
white\_axis \{bool\} \textendash{} If \sphinxtitleref{True} plot white axes (default: \{False\})
no\_legend \{bool\} \textendash{} If \sphinxtitleref{True} do not plot a legend (default: \{False\})
no\_star \{bool\} \textendash{} If \sphinxtitleref{True} do not plot a star (default: \{False\})

\end{description}

\end{fulllineitems}



\renewcommand{\indexname}{Python Module Index}
\begin{sphinxtheindex}
\let\bigletter\sphinxstyleindexlettergroup
\bigletter{c}
\item\relax\sphinxstyleindexentry{channel}\sphinxstyleindexpageref{index:\detokenize{module-channel}}
\indexspace
\bigletter{e}
\item\relax\sphinxstyleindexentry{environment}\sphinxstyleindexpageref{index:\detokenize{module-environment}}
\item\relax\sphinxstyleindexentry{eventcodes}\sphinxstyleindexpageref{index:\detokenize{module-eventcodes}}
\item\relax\sphinxstyleindexentry{events}\sphinxstyleindexpageref{index:\detokenize{module-events}}
\indexspace
\bigletter{f}
\item\relax\sphinxstyleindexentry{fish}\sphinxstyleindexpageref{index:\detokenize{module-fish}}
\indexspace
\bigletter{i}
\item\relax\sphinxstyleindexentry{interaction}\sphinxstyleindexpageref{index:\detokenize{module-interaction}}
\indexspace
\bigletter{o}
\item\relax\sphinxstyleindexentry{observer}\sphinxstyleindexpageref{index:\detokenize{module-observer}}
\indexspace
\bigletter{u}
\item\relax\sphinxstyleindexentry{utils}\sphinxstyleindexpageref{index:\detokenize{module-utils}}
\end{sphinxtheindex}

\renewcommand{\indexname}{Index}
\printindex
\end{document}